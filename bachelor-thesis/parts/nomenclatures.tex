\printnomenclature

\abbr{WiFi}{беспроводные компьютерные сети, это технология, позволяющая
создавать вычислительные сети, полностью соответствующие стандартам для обычных
проводных сетей, без использования кабельной проводки.}

\abbr{IEEE}{Институт Инженеров Электротехники и Радиоэлектроники.}

\abbr{Протокол передачи данных}{набор соглашений интерфейса логического
уровня, которые определяют обмен данными между различными программами.}

\abbr{WEP}{Wired Equivalent Privacy, протокол обеспечения безопасности
беспроводных сетей стандарта 802.11 WLAN.}

\abbr{Уязвимость}{недостаток в системе, используя который, можно
нарушить её целостность и вызвать неправильную работу.}

\abbr{RC4}{Rivest Cipher 4, это поточный шифр, широко применяющийся в
различных системах защиты информации в компьютерных сетях.}

\abbr{Поточный шифр}{это симметричный шифр, в котором каждый символ
открытого текста преобразуется в символ шифрованного текста в зависимости не
только от используемого ключа, но и от его расположения в потоке открытого
текста.}

\abbr{PRGA}{Pseudo-Random Generation Algorithm, генератор псевдослучайной
последовательности.}

\abbr{ARP}{Address Resolution Protocol, протокол определения адреса.}

\abbr{IV}{Initialization Vector, вектор инициализации.}

\abbr{WLAN}{Wireless Local Area Network, беспроводная локальная сеть.}

\abbr{KSA}{Key-Scheduling Algorithm, алгоритм ключевого расписания.}

\abbr{LLC}{Logical link control, подуровень управления логической связью в
компьютерных сетях.}
