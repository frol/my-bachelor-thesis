\chapter*{Введение}

На данный момент сети стандарта 802.11 WLAN (WiFi) широко используются в мире.
Технология WiFi получила широкое применение во многих современных предприятиях,
муниципальных учреждениях, школах, домах, квартирах, как более универсальная
альтернатива проводным локальным сетям. Для широкого распространения необходима
поддержка аппаратного обеспечения клиента, на даннный момент, поддержка
беспроводных сетей WiFi присутствует в любом ноутбуке, нетбуке, планшете и даже
в смартфоне.

Исходя из массового применения технологии WiFi, обеспечение безопасности в
стандарте 802.11 WLAN является актуальным вопросом на данный момент и существует
три основных протокола обеспечения безопасности: WEP, WPA, WPA2.

Целью данной работы является анализ работы сетей стандарта 802.11 WLAN,
защищённости WEP, протокола обеспечения безопасности, и известных атак на данный
протокол. Существуют активные и пассивные атаки на протокол WEP, которые обычно
применяются в комплексе. Атаки на протокол WEP возможны за счёт уязвимостей
поточного шифра RC4, а также уязвимостей в архитектуре самого протокола.

В разделе охрана труда будут исследованы вопросы условий труда, безопасности,
производсвтенной санитарии и пожарной безопасности в помещении
научно-исселедовательской лаборатории. Также будет произведён расчёт
коэффициента естественного освещения для города Харькова.
