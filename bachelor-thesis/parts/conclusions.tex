\chapter*{Выводы}

Беспроводные технологии получили широкое распространение, что привело к
необходимости обеспечивать безопасную работу в данных сетях. Таким образом, для
решения проблемы безопасности был разработан протокол WEP, однако, так как он
основывался на поточном алгоритме RC4, в котором были обнаружены уязвимости в
2001 году, а массовый выпуск беспроводного оборудования начался тоже в 2001
году, то протокол WEP был уязвим с самого начала.

Ввиду отсутствия альтернативных методов обеспечения безопасности он массово
применялся и получил большую популярность. Именно эта популярность и не даёт
совершить окончательный переход на протокол WPA2, который на данный момент не
имеет известных критических уязвимостей. Распространённости WEP также
способствует оборудование, которое не поддерживает WPA2. Как было отмечено
ранее, массовое применение беспроводных сетей отмечалось во многих современных
предприятиях, муниципальных учреждениях, школах, домах, квартирах, в этом
секторе очень мало людей подкованы в области информационной безопасности и даже
не подозревают об опасности использования открытых сетей или сетей с шифрованием
WEP. Таким образом, шифрование WEP будет использоваться ещё не один год.

Протокол WEP имеет два уязвимый аспекта, это архитектура протокола, которая даёт
возможность проводить активные атаки, и уязвимости алгоритма генерации ключей
поточного шифрования RC4, используемого в WEP, что позволяет проводить пассивные
атаки.

В первой части был рассмотрен стандарт 802.11 WLAN, процесс передачи пакетов,
обеспечение аутентификации и конфиденциальности.

Во второй части рассмотрен протокол WEP и алгоритм поточного шифрования RC4,
используемый в данном протоколе, и уязвимости RC4.

В третьей части исследованы возможные атаки на протокол WEP, описана
классификация атак и проведён анализ безопасности протокола WEP.

В четвёртой части представлена утилита WEPfrag, Данная утилита основана на атаке
с фрагментацией на протокол WEP. Данный метод позволяет получить псевдослучайную
последовательность (PRGA) и сформировать ARP-запрос, отправка которого позволит
создать трафик для получения большого количества векторов инициализации.
Большое количество векторов инициализации позволяет провести пассивную
статистическую атаку, в результате которой будет получен ключ WEP.

В разделе охраны труда был проведён анализ условий труда исследователя на своём
рабочем месте. Исследованы промышленная безопасность в помещении
научно-исследовательской лаборатории, производственная санитария, а также
производственная санитария в данном помещении. Так же был расчитан коэффициент
естественного освещения для города Харькова.

На данный момент наиболее эффективной пассивной атакой является атака PTW, а
среди активным атак --- атака с фрагментацией.
